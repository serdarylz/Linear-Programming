\documentclass[11pt]{article}

\usepackage{color}
\usepackage{listings}
\lstset{language=C++,
        breakindent=40pt,
        breaklines=true,
        keywordstyle=\color{blue},
        frame=single}
\usepackage{mathtools}
\usepackage[left=2cm,
            right=2cm]
            {geometry} 

\title{\textbf{PLAM}}
\author{Boulanger \'Edouard\\
		Chanut J\'er\'emy}
\date{}
\begin{document}

\maketitle

\section{Mod\'elisation objet}
Afin d'interfacer \textbf{GLPK} en C++, nous avons mod\'elis\'e les probl\`emes de PLNE avec trois classes: \emph{Variable}, \emph{Contrainte}, et \emph{Probl\`eme}.

\begin{table}[h]
\begin{center}
\begin{tabular}{|ll|}
\hline
 & \textbf{Problem} \\
\hline
- & pb: glp\_prob \\
- & rows: vector\verb|<|int\verb|>| \\
- & cols: vector\verb|<|int\verb|>| \\
- & values: vector\verb|<|double\verb|>| \\
\hline
+ & setOptimization(Optimization): void \\
+ & setConstraintsValues(Constraint, Variable, double): void \\
+ & solve(): vector\verb|<|double\verb|>| \\
\hline
\end{tabular}
\end{center}
\end{table}

\begin{table}[h]
\begin{center}
\begin{tabular}{|ll|}
\hline
 & \textbf{Variable} \\
\hline
- & lp: glp\_prob \\
- & j: int \\
\hline
+ & set\_name(string): void \\
+ & set\_bounds(int, double, double): void \\
+ & set\_coef(double): void \\
\hline
\end{tabular}
\end{center}
\end{table}

\begin{table}[h]
\begin{center}
\begin{tabular}{|ll|}
\hline
 & \textbf{Constraint} \\
\hline
- & lp: glp\_prob \\
- & i: int \\
\hline
+ & set\_name(string): void \\
+ & set\_bounds(int, double, double): void \\
\hline
\end{tabular}
\end{center}
\end{table}

\section{Probl\`eme des entrep\^ots}
$c_i$ repr\'esente la capacit\'e du site $i$. $n$ est le nombre de sites, $m$ le nombre d'entrep\^ots \`a construire. \emph{oui} est repr\'esent\'e par $1$, \emph{non} par $0$.

Nous avons comme variable un tableau de bool\'eens appel\'e $b$, de taille $n$. $b_i$ indique pour chaque site $i$ si oui ou non il est s\'electionn\'e pour y construire un entrep\^ot.

La fonction objectif \`a maximiser est donc: \[z=\sum_{i=1}^{n}b_i\times{}c_i\]

Pour s\'electionner $m$ sites , il faut la contrainte suivante: \[\sum_{i=1}^{n}b_i=m\]

La contrainte de distance, moins de 50 km entre deux entrep\^ots, est repr\'esent\'ee par un tableau de bool\'eens $d$ \`a deux dimensions, de taille $n\times{}n$. Chaque case $d_{i,j}$ indique si oui ou non les entrep\^ots $i$ et $j$ sont \`a moins de 50 km l'un de l'autre.

Ainsi pour s\'electionner $m$ sites en respectant la contrainte de distance, on a:

si $(d_{i,j} == 0)$

alors $b_{i}+b_{j} \leq 1$

%\[\sum_{\substack{i=1 \\ i<j}}^{n}b_i\times{}b_j\times{}d_{i,j}=m\]

\end{document}
